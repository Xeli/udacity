\documentclass[11pt]{article}
%Gummi|065|=)
\usepackage{amsmath}
\usepackage[hidelinks]{hyperref}
\title{\textbf{Machine learning Nanodegree\\ Capstone proposal}}
\author{Richard Deurwaarder}
\date{November 12th 2016}

\begin{document}

\maketitle

\section{Background}
This project is a Kaggle playground competition, called Dogs vs. Cats Redux: Kernels Edition\footnote{\url{https://www.kaggle.com/c/dogs-vs-cats-redux-kernels-edition}}. Originally this competition was run in 2013. With deep learning being more popular these days they have reintroduced the competition. It makes for a nice view into the  progression of Machine learning over the last tree years.
\section{The Problem}
The competition is about determining whether images are either cat or a dog by giving a probability to each image, 0 being cat, 1 being dog. The dataset is called the Asirra\footnote{Asirra: Animal Species Image Recognition for Restricting Access} data set. It consists of a training set with 25,000 images, half cats and the other half of dogs. And a test set of 12,500 images.
\section{Preprocessing the data}
I will need to preprocess the images because they're not all the same size. I will probably pad the smaller images with grey pixels to make them the same size. I might also grayscale the images, depending on how it will effect performance. Intuitively the color should not make much of a difference when determining if it's a cat or a dog, so it's a good way to reduce the complexity of the model.
\section{Solution}
The way I am hoping to solve this problem is by using a deep neural network with convolutional layers. Specifically by using the tensorflow framework. I'll use the following as loss function: 
\[ LogLoss = -\dfrac{1}{n}\sum\limits^{n}_{i=1}[y_i log(\hat{y}_i) + (1-y_i)log(1-\hat{y}_i)]
\]

This is the same loss funcion that Kaggle uses to evaluate the submissions of participants.

\section{Compare my model}
The competition already has some other submissions (200 at the time of writing) this will allow me to see what algorithms other people have used. And also see how my model performs versus the other participants.

\end{document}
